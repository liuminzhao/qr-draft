\documentclass[12pt]{article}
\usepackage{amsmath}
\usepackage[round]{natbib}
\usepackage{geometry}
%\usepackage{times}
\usepackage{graphicx}
%\usepackage{courier}
\usepackage{mathpazo}
\usepackage{mathrsfs}
\usepackage{bm}
\usepackage[colorlinks,linkcolor=red,anchorcolor=blue,citecolor=blue]{hyperref}
\usepackage{amsthm}
\usepackage{ulem}
\geometry{verbose,letterpaper,tmargin=1in,bmargin=.75in,lmargin=.75in,rmargin=1in}

\title{Bayesian Quantile Regression using a  Mixture of P\'{o}lya Tree Prior}
\date{\today}
\author{Minzhao Liu, Mike Daniels}


\newtheorem{thm}{Theorem}[subsection]
\newtheorem{deff}[thm]{Definition}
\newtheorem{rmk}[thm]{Remark}
% \newtheorem{prf}[thm]{Proof}
\newtheorem{cor}[thm]{Corollary}
\newtheorem{emp}[thm]{Example}
\newtheorem{lem}[thm]{Lemma}
\newtheorem{pps}[thm]{Proposition}

\newcommand{\polya}{P\'{o}lya}
\newcommand{\iid}{\stackrel{\text{i.i.d}}{\sim}}
\DeclareMathOperator{\pr}{p}
\DeclareMathOperator{\pt}{PT}


\begin{document}
% \setlength\parindent{0pt}

\maketitle{}

\section{Introduction}

Quantile regression is an attractive way of studying the relationship
between response and covariates when one (or several) quantiles are of
interest as compared to mean regression.  The dependence between upper
or lower quantiles of the response variable and the covariates are
expected to vary differentially relative to that of the average. This
is often of interest in econometrics, educational studies, biomedical
studies, and environment studies \citep{yu2001,buchinsky1994,
  buchinsky1998,he1998,koenker1999, wei2006, yu2003}.

A comprehensive review of quantile regression was presented in
\citep{koenker2005}.  Furthermore, mean regression provides less
information about the relationship of the average with linear
combination of covariates; quantile regression can offer a more
complete description of the conditional distribution of the response.

The traditional frequentist approach was proposed by
\citep{koenker1978} for a single quantile ($\tau$) with estimators
derived by minimizing a loss function. The popularity of this approach
is due to its computational efficiency by linear programming,
well-developed asymptotic properties, and straightforward extensions
to simultaneous quantile regression and random effect models. However,
asymptotic inference may not be accurate for small sample sizes.

Bayesian approaches offer exact inference. Motivated by the loss
(check) function, \citep{yu2001} proposed an asymmetric Laplace
distribution for the error term, such that maximizing the posterior
distribution is equivalent to minimizing the check function. Other
than parametric Bayesian approaches, some semiparametric methods have
been proposed for median regression. \citep{walker1999} used a diffuse
finite Polya Tree prior for the error term. \citep{kottas2001} modeled
the error by two families of median zero distribution using a mixture
Dirichlet process priors, which is very useful for unimodal error
distributions. \citep{hanson2002} adopted mixture of \polya{} Tree
prior on error term to make inference in regression model. They
illustrated the implementation on AFT model for the median survival
time, which showed robustness of \polya{} in terms of multimodality
and skewness. Other recent approaches include quantile pyramid priors,
mixture of Dirichlet process priors of multivariate normal
distributions and infinite mixture of Gaussian densities which put
quantile constraints on the residuals \citep{hjort2007,hjort2009,
  kottas2009, reich2010}.

Like the asymmetric Laplace distribution, all of the above methods are
single semiparametric quantile regression methods, which have some
limitations. The densities have their (restrictive) mode at the
quantile of interest, which is not appropriate when extreme quantiles
are being investigated. Other criticisms include crossed quantile
lines, monotonicity constraints, difficulty in making inference for
quantile regression parameter for an interval of $\tau$s. Joint
inference is poor in borrowing information through single quantile
regressions. It is not coherent to pool from every individual quantile
regression. Meanwhile, the sampling distribution of response for
$\tau_1$ might not be the same as that under quantile $\tau_2$. {\bf
  might be more specific about why the above issues are problems} {\it
  Minzhao: show mike tokdar2011}

In order to solve those problems, simultaneous linear quantile
regression have been proposed by \citep{tokdar2011}.  Another popular
approach is to assign a nonparametric model for the error term to
avoid the monotonicity problem (\citep{scaccia2003},
\citep{geweke2007}, \citep{taddy2010}).

We use \polya{} Tree (PT) priors in our approach. PT priors were
introduced decades ago (\citep{freedman1963}, \citep{fabius1964},
\citep{ferguson1974}) and Lavine extended them to \polya{} Tree models
(\citep{lavine1992, lavine1994}). The major advantage of \polya{} Tree
over Dirichlet process is that it can be absolutely continuous with
probability 1 and it can be easily tractable {\bf isn't the DP
  tractable too??} {\it Minzhao: DP is tractable too, here I only mean
  PT is better than DP in terms of continuous pdf.}. In a regression
context, \citep{walker1997, walker1999} assigned a finite \polya{}
Tree prior to the random effects in a generalized linear mixed
model. \citep{berger2001} used a mixture of \polya{} Tree comparing
data distribution coming from parametric distribution or mixture of
\polya{} Tree. {\bf this is not clear??}{\it Minzhao: used a \polya{}
  tree process to test the fit of data to a parametric model by
  embedding the parametric model in a nonparametric alternative and
  computing the Bayes factor of the parametric model to the
  nonparametric alternative.}  \citep{hanson2002} as mentioned earlier
modeled the error term as a mixture of \polya{} tree prior.

Multivariate regression is also possible with \polya{}
Tree. \citep{paddock1999, paddock2002} studied multivariate \polya{}
Tree in a k-dimensional hypercube. \citep{hanson2006} constructed a
general framework for multivariate random variable with a \polya{}
Tree distribution. \citep{jara2009} extended the multivariate mixture
of \polya{} Tree prior from Hanson with directional orthogonal matrix.
He also demonstrated how to fit a generalized mixed effect model by
modeling a multivariate random effects with multivariate mixture of
\polya{} Tree priors.

In this article, we present a Bayesian approach by adopting a mixture
of \polya{} Tree prior for the regression error term, and we account
for the change of quantile regression parameter via heterogeneity of
the error term. As a result, several quantile regression can be fit
simultaneously and there is a closed form for posterior quantile
regression parameter. Exact inference can be made through MCMC, and
our method avoids the problem of crossing quantile lines that occurs
in the traditional frequentist quantile regressions.

The rest of the paper is organized as follows. In Section 2, we
introduce the heterogeneity model and derive a closed form for
marginalized posterior quantile regression parameter with mixture of
\polya{} tree prior. We conduct some simulation studies in section 3
and use a real data example to illustrate our approach in Section
4. Finally, conclusions are presented in section 5.

\section{Model, Priors, and Computations}
\subsection{Heterogeneity Model}
Let $Y$ be a random variable with CDF $F$.  The $\tau$th quantile
of $Y$ is defined as
\begin{displaymath}
  Q_Y(\tau) = \underset{y}{\inf} \left\{ y: F(y) \ge \tau \right\}.
\end{displaymath}
If covariates $\bm{x_1, \ldots, x_n}$ are of interest,
then the quantile regression parameter satisfies this
condition:
\begin{displaymath}
  Q_Y(\tau) = \bm{X'\beta}(\tau),
\end{displaymath}
where $\bm{X}$ is the matrix of covariates including
an intercept (ie. $\bm{x_1}=
\bm{1}$).
If $F$ is continuous, then
$F(\bm{X'\beta}(\tau)) = \tau$, i.e., $\pr(Y \le \bm{X'\beta}(\tau)) =
\tau$.

Now, consider a location shift model,
\begin{displaymath}
  y_i = \bm{x}_i\beta + \epsilon_i,
\end{displaymath}
where $\epsilon_i \stackrel{\text{i.i.d}}{\sim} F_{\epsilon}$. Then, the $\tau$th
quantile regression parameter can be expressed as
\begin{equation}
  \label{eq:1}
  \bm{\beta}(\tau) = \bm{\beta} + F^{-1}_{\epsilon}(\tau) \bm{e}_1,
\end{equation}
where $\bm{e}_1 = [1, 0, \ldots, 0]^T$,  {\bf put the next little bit
  in an appendix} and $F^{-1}_{\epsilon}(\tau)$ is
the $\tau$th quantile for error $\epsilon$.
% \begin{align*}
%   \pr (Y \le \bm{X'\beta}(\tau)) & = \pr \left( \bm{x'\beta} + \epsilon
%     \le \bm{x'\beta} + F^{-1}_{\epsilon}(\tau) \right) \\
%   & = \pr (\epsilon \le F^{-1}_{\epsilon}(\tau)) \\
%   & = \tau.
% \end{align*}
{\bf up to here}

As we can see from equation (\ref{eq:1}), if the model is homogeneous
, i.e., i.i.d case, then for different quantiles $\tau$, the
corresponding quantile regression parameters only vary in the
first component, the intercept. The rest of the quantile regression
parameters stay the same. Therefore, quantile lines for different
quantiles are parallel to each other.
{\bf i would put figure 1 in the dissertation, but not the paper}
% Figure~\ref{fig:homo} for a single quantile regression as example.

% \begin{figure}[h]
%   \centerline{\includegraphics[scale=0.4]{homo}}
%   \caption[]{\label{fig:homo} A Single Quantile Regression: True
%     distribution is $1+3x + \epsilon$, where $\epsilon \sim
%     \mathrm{N}(0,1)$. The scatter plot contains 500 samples and true quantile
%     lines for quantiles $\tau=0.1,0.25,0.5,0.75,0.9$ are shown in the
%     left panel. The middle and right panel show how the true quantile
%     regression parameter, intercept and slope, change with different
%     quantiles in a homogeneous model. Only the intercept changes in such
%     a case ($\beta_0+\gamma_0 F^{-1}_{\tau}(\epsilon)$) , while the slopes
%     stay the same.}
%   \label{fig:homo}
% \end{figure}

Now, consider the heterogeneous linear regression model  from \citep{he1998}
\begin{equation}\label{eq:8}
  y_i = \bm{x}_i'\bm{\beta} + (\bm{x}_i'\bm{\gamma}) \epsilon_i,
\end{equation}
{\bf need an equation \# for this one}
where $\bm{x_i'\gamma}$ is positive  for all
$i$. Under this model, the $\tau$th quantile regression parameter is
\begin{equation}\label{eq:2}
  \bm{\beta}(\tau) = \bm{\beta} + F^{-1}_{\epsilon}(\tau) \bm{\gamma},
\end{equation}
{\bf put this little 'proof' in the appendix too} %since
% \begin{align*}
%   \pr (Y \le \bm{x'\beta}(\tau)) & = \pr \left( \bm{x'\beta} +
%     (\bm{x}'\gamma) \epsilon \le \bm{x'\beta} + (\bm{x'\gamma})
%     F^{-1}_{\epsilon}(\tau) \right) \\
%   & = \pr \left( (\bm{x'\gamma}) \epsilon \le  (\bm{x'\gamma})
%     F^{-1}_{\epsilon}(\tau)  \right)\\
%   & = \pr (\epsilon \le F^{-1}_{\epsilon}(\tau)) \\
%   & = \tau .
% \end{align*}
{\bf figure for dissertation}
% Figure~\ref{fig:hetero} shows a single $\tau$th quantile regression under
% heterogeneous linear model.

% \begin{figure}[h]
%   \centerline{\includegraphics[scale=0.4]{heter}}
%   \caption[]{ \label{fig:heter} A Single Quantile Regression: True
%     distribution is $1+3x + (1+0.5x)\epsilon$, where $\epsilon \sim
%     \mathrm{N}(0,1)$. The scatter plot of 500 samples and true quantile
%     lines for quantiles $\tau=0.1,0.25,0.5,0.75,0.9$ are shown in the
%     left panel. The middle and right panel show how the true quantile
%     regression parameter, intercept and slope, change with different
%     quantiles in this heterogeneous model. The slope is now changes
%     for different quantiles via the heterogeneity parameter $\bm{\gamma}
%     \neq \bm{e}_1$.  }
%   \label{fig:hetero}
% \end{figure}

% Clearly,
Quantile lines
are no longer parallel
under the heterogeneous linear model which adds considerably more
flexibility in the model.

{\bf this paragraph should probably be merged with material in the
  introduction}
{\it Minzhao: merge to introduction}
Traditional single quantile regression make different assumptions on
the error term. The frequentist approach of \citep{koenker1978} does not
assign distributions for the residual, and uses linear
programming technique to minimize the check function $\sum_{i=1}^n
\rho_{\tau}(y_i - \bm{x_i'\beta})$, where $\rho_{\tau}(\epsilon) =
\epsilon (\tau- \mathrm{I}(\epsilon < 0))$.
Some Bayesian
approaches specify the error distribution as an asymmetric Laplace
distribution (\citep{yu2001}), or Dirichlet process prior
(\citep{kottas2001}, \citep{kottas2009},  \citep{taddy2010}) or
\polya{} tree prior (\citep{walker1999}, \citep{hanson2002}). \citep{reich2010}
uses an infinite mixture of Gaussian densities on the
residual. However, all these densities keep their
restrictive mode at the quantile of interest, i.e., in equation
(\ref{eq:2}), $\bm{\beta}(\tau) \equiv \beta$. Other limitations exist
as well such as crossing quantile lines, monotonicity constraints,
non-coherent joint quantile regression inference from each single
quantile regression. The sampling distribution of response in
$\tau_1$th quantile regression is not even the same as that in
$\tau_2$th quantile regression.

We use a mixture of \polya{} tree prior for the error term in (\#) and
derive a
closed form for posterior quantile regression parameter in
(\ref{eq:2}).
Since \polya{}
tree are a very flexible way to model the unknown distribution, our
approach makes fewer assumptions.
Exact
inference can be made through MCMC and functional of posterior
samples. The next subsection briefly reviews the \polya{} tree priors and
their relevant properties.

\subsection{\polya{} Tree}
\citep{lavine1992, lavine1994} and \citep{mauldin1992} developed theory for
\polya{} tress priors as a generalization of the Dirichlet
process (\citep{ferguson1974}). Denote $E=\{0,1\}$ and $E^m$ as the m-fold
product of $E$, $E^0= \emptyset$, $E^{*} = \cup_0^{\infty} E^m$ and $\Omega$ be a separable
measurable space , $\pi_0 = \Omega$, $\Pi= \{ \pi_m: m=0,1, \ldots \}
$ be a separating binary tree of partitions of $\Omega$. In addition,
define $B_{\emptyset} = \Omega$ and $\forall \epsilon=\epsilon_1\cdots
\epsilon_m \in E^{*}$, $B_{\epsilon 0}$ and $B_{\epsilon 1}$ are the
two partition of $B_{\epsilon}$.
\begin{deff}
  A random probability measure $G$ on $(\Omega, \mathcal{F})$ is said to
  have a \polya{} tree distribution, or a \polya{} tree prior with
  parameter $(\Pi, \mathcal{A})$, written as $G|\Pi, \mathcal{A} \sim
  \pt (\Pi, \mathcal{A})$, if there exist nonnegative numbers
  $\mathcal{A}= \left\{ \alpha_{\epsilon}, \epsilon \in E^{*} \right\}$
  and random vectors $\mathcal{Y} = \left\{ Y_{\epsilon} : \epsilon \in
    E^{*} \right\}$ such that the following hold:
  \begin{enumerate}
  \item\label{item:1} all the random variables in $\mathcal{Y}$ are independent;
  \item $Y_{\epsilon}= (Y_{\epsilon 0} , Y_{\epsilon 1}) \sim
    \mathrm{Dirichlet}(\alpha_{\epsilon 0 }, \alpha_{\epsilon 1}),
    \forall \epsilon \in E^{*}$;
  \item $\forall m=1,2, \ldots$, and $\forall \epsilon \in E^{*},
    G(B_{\epsilon_{1}, \ldots, \epsilon_m}) = \prod_{j=1}^m Y_{\epsilon_1
      \cdots \epsilon_j}$.
  \end{enumerate}
\end{deff}

\subsubsection{\polya{} Tree Parameters}
There are two parameters in the \polya{} tree distribution $(\Pi,
\mathcal{A})$. {\it Minzhao: If a \polya{} tree is centered around a
  pre-specified distribution $G_0$, which is called the baseline
  measure, } the $\mathcal{A}$ family determines how much $G$ can
deviate from $G_0$. {\it \sout{ , the baseline measure}} {\bf need to define $G_0$ as
  the base measure before this, $\Pi(G_0)$}. \citep{ferguson1974} pointed out
$\alpha_{\epsilon = 1} $ yields a $G$ that is continuous singular with
probability 1, and $\alpha_{\epsilon_1, \ldots, \epsilon_m} = m^2$
yields $G$ that is absolutely continuous with probability 1. \citep{walker1999}
and \citep{paddock1999} considered $\alpha_{\epsilon_1,
  \ldots, \epsilon_m} = cm^2$, where $c > 0$. \citep{berger2001}
considered $\alpha_{\epsilon_1, \ldots, \epsilon_m} = c
\rho(m)$. In general, any $\rho(m) $ such that $\sum_{m=1}^{\infty}
\rho(m)^{-1} < \infty$ guarantees $G$ to be absolutely continuous. In
our case, we adopt $\alpha_{\epsilon_1, \ldots, \epsilon_m} = cm^2$.

As to the partition parameter $\Pi$, the canonical way of constructing
a \polya{} tree distribution $G$ centering on $G_0$, a continuous CDF
is to choose $B_0 = G^{-1}_0 ([0, 1/2]), B_1 = G^{-1}_0 ((1/2,1])$,
such that $G(B_0) = G(B_1)= 1/2$. Furthermore, for all $\epsilon \in
E^{*}$, choose $B_{\epsilon 0 }$ and $B_{\epsilon 1}$ to satisfy
$G(B_{\epsilon 0 } |B_{\epsilon} ) = G(B_{\epsilon 1} | B_{\epsilon})
= 1/2 $, then any choice of $\mathcal{A} $ makes $G$ coincide with
$G_0$. A simple example is to choose $B_{\epsilon 0} $ and
$B_{\epsilon 1}$ in level $m$ by setting them as $G^{-1}_0 \left(
  (k/2^m, (k+1)/2^m] \right)$ , for $k=0, \ldots, 2^m-1$.

\subsubsection{Some properties of \polya{} Tree}
Suppose $G \sim \pt (\Pi, \mathcal{A})$ is a random probability
measure and $\epsilon_1, \epsilon_2, \ldots$ are a random sample from $G$.

\begin{deff}[Expectation of \polya{} Tree]
  $F= E(G)$ as a probability measure is defined by $F(B) = E(G(B)),
  \forall B \in \mathcal{B}$. By the definition of \polya{} tree, for any
  $\epsilon \in E^{*}$,
  \begin{displaymath}
    F(B_{\epsilon})  = E(G(B_{\epsilon})) = \prod_{j=1}^m
    \frac{\alpha_{\epsilon_1, \ldots, \epsilon_j}}{\alpha_{\epsilon_1,
        \ldots, \epsilon_{j-1},0} + \alpha_{\epsilon_1, \ldots, \epsilon_{j-1},1}}.
  \end{displaymath}
\end{deff}

\begin{rmk}
  If $G$ is constructed based on baseline measure $G_0$ and we set
  $\alpha_{\epsilon_1, \ldots, \epsilon_m} = cm^2 $,
  $\epsilon_{\epsilon_0 }= \alpha_{\epsilon_1}$, then $\forall B \in
  \mathcal{B}, F(B) = G_0(B)$; thus, $F=G_0$, if there is no
  data.
\end{rmk}

\begin{deff}[Density Function]
  Suppose $F=E(G), G|\Pi, \mathcal{A} \sim \pt (\Pi, \mathcal{A})$,
  where $G_0 $ is the baseline measure. Then, using the canonical
  construction, $F=G_0$ (as shown above), the density function is
  \begin{equation}\label{eq:3}
    f(x) = \left[ \prod_{j=1}^m \frac{ \alpha_{\epsilon_1, \ldots,
          \epsilon_j}(x)}{\alpha_{\epsilon_1, \ldots, \epsilon_{j-1},0}(x)
        + \alpha_{\epsilon_1, \ldots, \epsilon_{j-1},1}(x)} \right] 2^{m } g_0(x),
  \end{equation}
  where $g_0$ is the pdf of $G_0$.
\end{deff}

\begin{rmk}
  When using the canonical construction with no data,
  $\alpha_{\epsilon_0 } = \alpha_{\epsilon_1}$, equation (\ref{eq:3})
  simplifies to
  \begin{displaymath}
    f(x) = g_0(x).
  \end{displaymath}
\end{rmk}

\begin{rmk}[Conjugacy]
{\bf should use y's instead of x's here??}  If $y_1, \ldots, y_n | G \sim G, G|\Pi, \mathcal{A} \sim \pt (\Pi,
  \mathcal{A})$, then $G|y_1, \ldots, y_n , \Pi, \mathcal{A} \sim \pt
  (\Pi, \mathcal{A}^{*})$, where in $\mathcal{A}^{*}, \forall \epsilon
  \in E^{*}$,
  \begin{displaymath}
    \alpha_{\epsilon}^{*} = \alpha_{\epsilon} + n_{\epsilon}(y_1, \ldots, y_n),
  \end{displaymath}
  where $n_{\epsilon}(y_1, \ldots, y_n)$ indicates the count how many
  samples of $y_1, \ldots, y_n$ drop in $B_{\epsilon}$.
\end{rmk}

\subsubsection{Mixture of \polya{} Tree}
The behavior of a single \polya{} tree highly depends on how the
partition is separated. A random probability measure $G_\theta$ is
said to be a mixture of \polya{} tree if there exists a random
variable $\theta$ with distribution $h_{\theta}$ , and \polya{} tree
parameters $(\Pi_{\theta}, \mathcal{A}_{\theta})$ such that
$G_{\theta} | \theta=\theta \sim \pt (\Pi^{\theta},
\mathcal{A}^{\theta})$.

\begin{emp}
  Suppose $G_0 = \mathrm{N}(\mu, \sigma^2)$ is the baseline measure.
  For $\epsilon \in E^{*}, \alpha_{\epsilon_m} = cm^2 $,
  $\bm{\theta}= (\mu, \sigma^2, c)$ is the mixing index and the
  distribution on $\Theta = (\mu, \sigma^2, c) $ is the mixing
  distribution.
\end{emp}
With the mixture of \polya{} tree, the influence of the partition
is lessened. Thus, inference will not be affected greatly by a single
\polya{} tree distribution.

\subsubsection{Predictive Error Density, Cumulative Density Function
  and Quantiles}
Suppose $G_{\theta} = \mathrm{N}(0, \sigma^2)$ is the baseline
measure, $g_0(x) = \phi(x; 0, \sigma^2)$ is the density
function. $\Pi^{\theta}$ is defined as
\begin{displaymath}
  B^{\theta}_{\epsilon_1, \ldots, \epsilon_m} = \left( G^{-1}_{\theta}
    \left( \frac{k}{2^m} \right) , G^{-1}_{\theta}\left( \frac{k+1}{2^m} \right) \right),
\end{displaymath}
where $k$ is the index of partition $\epsilon_1, \ldots, \epsilon_m$
in level $m$. $\mathcal{A}^c$ is defined as
\begin{displaymath}
  \alpha_{\epsilon_1, \ldots, \epsilon_m} = cm^2.
\end{displaymath}
Therefore, the error model is
\begin{align*}
  x_1, \ldots, x_n |G_{\theta} & \iid G, \\
  G|\Pi^{\theta}, \mathcal{A}^{c} & \sim \pt (\Pi^{\theta},
  \mathcal{A}^{c}).
\end{align*}

The predictive density function of $X|x_1, \ldots, x_n, \theta$ ,
marginalizing out $G$ , is
\begin{equation}
  \label{eq:4}
  f_x^{\theta} (x|x_1, \ldots, x_n)  = \lim_{m \to \infty} \left[
    \prod_{j=2}^m \frac{cj^2 + n_{\epsilon_1 \cdots \epsilon_j(x) }(x_1 , \ldots, x_n)}{2cj^2
      + n_{\epsilon_1 \cdots \epsilon_{j-1}(x)}(x_1, \ldots, x_n)}
  \right]2^{m-1} g_0(x),
\end{equation}
where $n_{\epsilon_1 \cdots \epsilon_j(x) }(x_1 , \ldots, x_n)$
denotes the number of observations $x_1, \ldots, x_n$ dropping in the
slot $\epsilon_1 \cdots \epsilon_j$ where $x$ stays in the level
$j$. Notice that, if  we restrict the first level weight as
$\alpha_0=\alpha_1=1$, then we only need to update levels other than
the first level.

\begin{rmk}[The predictive density for Finite \polya{} Tree]
  In practice, a finite $M$ level \polya{} Tree is usually adopted to
  approximate the full \polya{} tree, in which, only up to $M$ levels are
  updated. The corresponding predictive density becomes
  \begin{equation}
    \label{eq:5}
    f_x^{\theta, M} (x|x_1, \ldots, x_n)  =  \left[
      \prod_{j=2}^M \frac{cj^2 + n_{\epsilon_1 \cdots \epsilon_j(x) }(x_1 , \ldots, x_n)}{2cj^2
        + n_{\epsilon_1 \cdots \epsilon_{j-1}(x)}(x_1, \ldots, x_n)}
    \right]2^{M-1} g_0(x).
  \end{equation}
  The rule of thumb for choosing $M$ is to set $M=\log_2n$, where $n$ is
  the sample size.
\end{rmk}

\noindent \citep{hanson2002} showed the approximation to (4) given in
(\ref{eq:5}) is exact for $M$ large enough.  We now derive the
predictive cdf and the predictive quantile(s).

\begin{thm}
  Based on the predictive density function (\ref{eq:5}) of a
  finite \polya{} tree distribution,
  the predictive cumulative density function is
  \begin{equation}
    \label{eq:6}
    F^{\theta,M}_X(x|x_1, \ldots, x_n) = \sum_{i=1}^{N-1} P_{i} + P_N
    \left( G_{\theta}(x)2^M -(N-1) \right),
  \end{equation}
  where
  \begin{align*}
    P_i &= \frac{1}{2} \left\{\prod_{j=2}^M \frac{cj^2 + n_{j,\lceil i2^{j-M}
          \rceil}(x_1, \ldots, x_n)}{2cj^2 + n_{j-1,\lceil
          i2^{j-1-M} \rceil}(x_{1 },\ldots, x_n)} \right\} \mbox{ and}\\
    N & = \left[ 2^{M } G_{\theta}(x)   +1\right],
  \end{align*}
  in which $n_{j,\lceil i2^{j-M}
    \rceil}(x_1, \ldots, x_n)$ denotes the number of observations $x_1,
  \ldots, x_n$ in the $\lceil i2^{j-M}
  \rceil$ slot at level $j$, $\lceil \cdot
  \rceil$ is the ceiling function, and $[ \cdot ]$ is the floor function.
\end{thm}

\begin{proof}
  \begin{align*}
    F^{\theta,M}_X(x| x_1, \ldots, x_n) & = \int_{-\infty}^x
    f_x^{\theta,M} (x|x_1, \ldots, x_n) dx \\
    & = \int_{-\infty}^x \left[
      \prod_{j=2}^M \frac{cj^2 + n_{\epsilon_1 \cdots \epsilon_j(x) }(x_1 , \ldots, x_n)}{2cj^2
        + n_{\epsilon_1 \cdots \epsilon_{j-1}(x)}(x_1, \ldots, x_n)}
    \right]2^{M-1} g_\theta(x) dx \\
    & =  \sum_{i=1}^{N-1} \left[ \prod_{j=2}^M \frac{cj^2 + n_{j, \lceil i2^{j-M}
          \rceil}(x_1,
        \ldots, x_n)}{2cj^2 + n_{j-1, \lceil i2^{j-1-M}
          \rceil}(x_1, \ldots, x_n)} 2^{M-1}
      \int_{\epsilon_{M,i}} g_{\theta}(x) dx \right] \\
    &+
    \int_{G^{-1}_{\theta}((N-1)/2^M)}^x \left[ \prod_{j=2}^M \frac{cj^2 + n_{j, \lceil N2^{j-M}
          \rceil}(x_1,
        \ldots, x_n)}{2cj^2 + n_{j-1, \lceil N2^{j-1-M}
          \rceil}(x_1, \ldots, x_n)}\right] 2^{M-1}
    g_{\theta}(x) dx \\
    & = \sum_{i=1}^{N-1} P_i + P_N 2^M \left( G_{\theta}(x) -
      G_{\theta}(G_{\theta}^{-1}\left( \frac{N-1}{2^M} \right)\right)\\
    & = \sum_{i=1}^{N-1}P_i + P_N \left( G_{\theta}(x) 2^M - (N-1) \right),
  \end{align*}
  where $\epsilon_{M,i}$ is the $i$th partition in level $M$.
\end{proof}

\begin{thm}
  The posterior predictive quantile of finite \polya{} tree
  distribution is
  \begin{equation}
    \label{eq:7}
    Q^{\theta, M}_{X|x_1, \ldots, x_n}(\tau) = G^{-1}_{\theta} \left[
      \frac{\tau- \sum_{i=1}^N P_i + N P_N}{2^M P_N} \right],
  \end{equation}
  where $N$ satisfies $ \sum_{i=1}^{N-1} P_i < \tau \le \sum_{i=1}^N P_i$.
\end{thm}

\begin{proof}
  From equation (\ref{eq:6}),
  \begin{align*}
    \tau = F^{\theta,M}_X(x|x_1, \ldots, x_n) &= \sum_{i=1}^{N-1} P_{i} + P_N
    \left( G_{\theta}(x)2^M -(N-1) \right) \\
    \Rightarrow G_{\theta}(x) &= \frac{\tau - \sum_{i=1}^NP_i +
      NP_N}{2^MP_N} \\
    x & = G_{\theta}^{-1} \left[\frac{\tau - \sum_{i=1}^NP_i +
        NP_N}{2^MP_N}  \right].
  \end{align*}
\end{proof}
{\bf Now explicitly state result (corollary) for closed form for (2)}
Now the explicit form for quantile regression coefficients in equation
(\ref{eq:2}) becomes:
\begin{equation}
\label{eq:9}
  \bm{\beta}(\tau) = \bm{\beta} + \bm{\gamma}G_{\theta}^{-1}
  \left[\frac{\tau - \sum_{i=1}^NP_i +
        NP_N}{2^MP_N}  \right] ,
\end{equation}
where $P_i$ and $N$ are the notations in equation (\ref{eq:6}) and (\ref{eq:7}).

\subsection{Fully Bayesian Quantile Regression Specification with Mixture of \polya{}
  Tree Priors}
The full Bayesian specification of our quantile regression is given as
follows,
\begin{align*}
  y_i& = \bm{x_i'\beta} + (\bm{x_i'\gamma}) \epsilon_{i}, i = 1, \ldots,
  n \\
  \epsilon_i |G_{\theta} & \iid G_{\theta} \\
  G_{\theta}|\Pi^{\theta}, \mathcal{A}^{\theta} & \sim \pt
  (\Pi^{\theta}, \mathcal{A}^{\theta}) \\
  \bm{\theta} = (\sigma^2, c) & \sim \pi_{\theta}(\theta) \\
  \bm{\beta} & \sim \pi_{\beta}(\beta)\\
  \bm{\gamma} &\sim \pi_{\gamma}(\gamma).
\end{align*}
In order to not confound the location parameter, $\epsilon_i $ or $G$
is set to have median 0 by fixing $\alpha_0=\alpha_1 = 1$. For the
similar reasons,
the first component of $\bm{\gamma}$ is fixed
at 1.

The posterior distribution of $(\bm{\beta},
\bm{\gamma}, \sigma^2, c)$ is given as
\begin{align*}
  \pr(\bm{\beta}, \bm{\gamma}, \sigma^2, c|\bm{Y}) & \propto L(\bm{Y}|
  \bm{\beta}, \bm{\gamma}, \sigma^2, c) \pi_{\beta}(\beta)
  \pi_{\gamma}(\gamma) \pi_{\sigma^2}(\sigma^2) \pi_c(c) \\
  & = \frac{1}{\prod_{i=1}^n (\bm{x_i'\gamma})} \pr \left( \epsilon_1,
    \ldots, \epsilon_n | \bm{\beta}, \bm{\gamma}, \sigma^2, c\right)
  \pi_{\beta}(\beta)
  \pi_{\gamma}(\gamma) \pi_{\sigma^2}(\sigma^2) \pi_c(c) \\
  & = \frac{1}{\prod_{i=1}^n (\bm{x_i'\gamma})} \pr \left(\epsilon_n| \epsilon_1,
    \ldots, \epsilon_{n-1} , \bm{\beta}, \bm{\gamma}, \sigma^2, c\right)
  \cdots  \pr \left(\epsilon_2| \epsilon_1, \bm{\beta}, \bm{\gamma},
    \sigma^2, c\right)  \pr \left(\epsilon_1| \bm{\beta}, \bm{\gamma},
    \sigma^2, c\right)\\
  & \qquad
  \pi_{\bm{\beta}}(\bm{\beta})
  \pi_{\bm{\gamma}}(\bm{\gamma}) \pi_{\sigma^2}(\sigma^2) \pi_c(c) .\\
\end{align*}
{\bf Need to discuss choices of priors here}

Usually priors for parameters $(\bm{\beta}, \bm{\gamma})$
could be diffused p-dimensional normal distribution . Diffused gamma
distribution could be chosen as priors for $\sigma^2$ and $c$. For
shrinkage model, spike priors could be adopted to shrink the parameter
estimates to pre-specified values. In addition, spike priors can also
help variable selection in Bayesian model and shrink heterogeneity
parameters to zero to find homogeneous model. Moreover, spike and slab
priors can help to accommodate zero-inflated situation and research
hypothesis in variable selection.

\section{Simulations}
We
conduct a simulation study to compare our approach with other existing
methods, specifically, the 'rq' function in the 'quantreg' package in
R
(the standard frequentist quantile regression method) and
flexible Bayesian quantile regression approach by Reich
('BQR').
We compare quantile regression approaches
for both homogeneous and heterogeneous models.

\subsection{Design}
We generated data from the following 3 models,
\begin{itemize}
\item [M1:] $y_i = 1 + x_{i1}\beta_1 + x_{i2}\beta_2 + \epsilon_{1i}$,
\item [M2:] $y_i = 1 + x_{i1}\beta_1 + x_{i2}\beta_2 + \epsilon_{2i}$,
\item [M3:]  $y_i = 1 + x_{i1}\beta_1 + x_{i2}\beta_2 +
  (1-0.5x_{i1}+0.5x_{i2}) \epsilon_{2i}$,
\end{itemize}
where $x_{i1} , x_{i2} \stackrel{\text{iid}}{\sim} \mathrm{N}(0,1),
\epsilon_{1i} \sim \mathrm{N}(0,1), \epsilon_{2i}
\stackrel{\text{iid}}{\sim} 0.5 \times \mathrm{N}(-2,1) +
0.5 \times \mathrm{N}(2,1)
$. All covariates and error terms are mutually independent. All
coefficients are set to be 1. For each model, we generate 100 data
sets with the sample size $n=100$. $\tau=0.5, 0.9$ are the quantiles
of interest.

Each simulated data set is analyzed using the three methods. For the proposed
Bayesian linear quantile regression with \polya{} Tree prior (PT),
we adopt the following prior specifications: $\mu_{\beta} = (0,0,0)^T,
\Sigma_{\beta}=\text{diag} (\sqrt{1000},\sqrt{1000},\sqrt{1000}), $
$\mu_{\gamma} = (0,0,0)^T$,
$\Sigma_{\gamma}=\text{diag} (\sqrt{1000},\sqrt{1000},\sqrt{1000}) $,
and $\tau=(0.01, 0.01)$. A partial \polya{} tree with $M=6$ levels was adopted
in the model. For Monte Carlo Markov chain parameter, 220,000
iterations of a single Markov chain were used, during which, 10,000
samples were saved through every 20 steps after a burn-in period of
20,000 samples. {\bf can comment on time??} {\it Minzhao: 330 seconds
  for one simulation
  under Linux 2.6.32 kernel, x86\_64 machine, 8g memory, 4 core
  2.66GHz cpu.} Acceptance rates for $\bm{\beta}$ and $\bm{\gamma}$ candidates
during the adaptive Metropolis-Hastings algorithm were set to approach
25\%.
We also use the method proposed by Reich (BQR),
which conducts a single $\tau$
quantile regression for linear model and assigns an infinite mixture
of Gaussian densities for the error term and the
standard frequentist quantile regression approach , 'rq' function in
the 'quantreg' package in R (RQ).

Methods are evaluated  based on mean squared error:
\begin{displaymath}
  \text{MSE}  = \frac{1}{p} \sum_{j=1}^p (\hat{\beta}_j(\tau) -
  \beta_j(\tau))^2,
\end{displaymath}
where $p$ is the number of covariates except the intercept (so here,
$p=2$). $\beta_j(\tau)$ is the
$j-$component of the true quantile regression
parameters. $\hat{\beta}_j(\tau)$ is the  $j$th component of estimated
quantile regression parameters. (we use the posterior median for
the Bayesian approaches).

\begin{center}
  \begin{table}[h]
    \centering
    \caption[]{ Mean squared error (reported as 100*average) and standard
      error (reported as 100*standard error) for each
      quantile regression method.  {\bf identify abbreviations in
        table in the table caption here} {\it Minzhao: The three
        columns ('rq', 'BQR',
        'PT') stand for frequentist method (rq) function from
        'quantreg' R package, flexible Bayesian method by Reich , and
        our Bayesian approach using \polya{} tree separately.}}
    \vspace{4mm}
    \begin{tabular}[tb]{l|l|lll}
      \hline
      Model & quantile & rq         & BQR        & PT          \\
      \hline
      M1    & 0.5      & 1.39(0.13) & 0.96(0.10) & 0.96(0.09)  \\
      M2    &          & 17.2(1.48) & 4.09(0.5)  & 1.89(0.26)  \\
      M3    &          & 95.4(6.69) & 16.5(1.90) & 6.29(0.86)  \\
      \hline
      M1    & 0.9      & 2.35(0.26) & 1.96(0.25) & 1.79(0.17)  \\
      M2    &          & 5.73(1.03) & 4.32(0.54) & 3.83(0.49)  \\
      M3    &          & 25.1(2.66) & 12.4(1.44) & 14.06(1.39) \\
      \hline
    \end{tabular}
    \label{tab:1}
  \end{table}
\end{center}

\subsection{Results}
The simulation results are shown in Table~\ref{tab:1}. The proposed
Bayesian quantile regression method with \polya{} tree prior (PT)
does very well (in terms of MSE) relative to Reich's method (BQR)
and traditional frequentist approach
(rq).
The differences becomes quite large in
the  non-unimodal case (M2) and heterogeneous model (M3).

In Model 2 and Model 3 with $\tau=0.5$, where the error is
distributed as a bimodal distribution (mixture of normal
distributions), the rq method performs poorly in terms of MSE
since the mode of the error is no longer the quantile of
interest. In contrast, our method (PT) is not impacted by lack of
unimodality and heterogeneity and provides more information
for the relationship between responses and covariates. In Model 3
with $\tau=0.9$, Reich's method (BQR) outperforms our approach (PT),
since in his model, the error is assigned an infinite mixture of
normal distribution with mode at $\tau=0.9$, which is very close to
the true distribution. Less information is available from our
approach to detect the shape at a particular quantile of the
distribution since there are few observations at extreme quantiles.

\section{Analysis of the Tours Data}
In this section, we apply our Bayesian quantile
regression approach to examine the quantiles of 6 month weight loss from a recent
weight management study (Perri et al., 2008?).  In particular, we are
interested in the effects of age and race.
The response of interest is the  weight loss
from  baseline to  6 months. The age of
the subjects ranged from 50 to 75, and there were 43 people with race
classifed as
black (race=1) and 181 people, white (race=2).
Our goal is to determine how  the percentiles of weight change
are affected by their age and race.

\begin{figure}[h]
  \begin{minipage}{0.5\linewidth}
    \centerline{\includegraphics[scale=0.4]{../bqrpt/results/change-age}}
  \end{minipage}
  \begin{minipage}{0.5\linewidth}
    \centerline{\includegraphics[scale=0.4]{../bqrpt/results/change-race1}}
  \end{minipage}

  \caption[]{\label{fig:tours} Scatterplots of weight change vs age and
    Boxplots of weight loss for each race. On the left figure, Red
    solid line is the   fitted line from regular mean regression model
    with one covariate   'age', and blue dashed line is the fitted
    lowess line for model.  The
    boxplots use the default settings: (0.75, 0.5, 0.25) quantile for
    box and $Q1-1.5IQR$ for lower whisker and $Q3+1.5IQR$ for upper
    whisker. }
\end{figure}

Figure 3 shows  weight loss vs age and the boxplots by race.
There does not appear to be
heterogeneity of the response errors. Weight loss though does appear
to be a bit right
skewed.

For numerical stability,
age was centered and standardized.


\begin{table}[h]
  \caption[]{\label{tab:tours} 95\% credible (confidence) intervals for
    tours data quantile regression parameter for the
    traditional frequentist approach (QReg) and the proposed PT approach.}
  \vspace{4mm}

  \centering
  \begin{tabular}{r|r|rrrr}
    \hline
              & $\tau$ & PT             & QReg           & PT Estimates
              & QReg Estimates                                           \\
    \hline
    Intercept & 0.5    & (9.73, 11.20)  & (9.31, 10.80)  & 10.38 & 10.30 \\
    Age       &        & (-1.02, 0.07)  & (-1.27, 0.17)  & -0.41 & -0.69 \\
    Race 1    &        & (-5.28, -2.33) & (-5.46, -2.42) & -3.88 & -3.53 \\
    \hline
    Intercept & 0.9    & (16.76, 18.16) & (16.64, 18.42) & 17.41 & 17.38 \\
    Age       &        & (-1.28, 0.50)  & (-1.93, -0.06) & -0.24 & -0.86 \\
    Race 1    &        & (-6.70, -2.41) & (-6.85, -2.48) & -4.77 & -6.08 \\
    \hline
  \end{tabular}
\end{table}

Results appear in Table~\ref{tab:tours}.
Both methods  show the median
and 90\% percentile for weight loss are significantly  affected by
race, which can be interpreted as whites generally tend to
lose more weight than blacks and furthermore, this differential
becomes larger when comparing the most successful (highest) weight losers (90\%
percentile). The results for 'age'
parameter in 90\% quantile regression differ between the two
approaches.  Our approach
indicates the relationship between weight loss and age in terms of
90\% percentile is not significant, while the traditional frequentist
method shows the age did affect weight loss.
{\bf also more specifically comment on the large differences in
  estimates between the .5 and .9 quantile and between the qreg and pt
  approaches}
{\it Minzhao: Estimates for intercept in median and 90\% percentile
  model from both methods are almost the same. Other coefficients
  estimates ('age' and 'race') keep the same sign. However, when
  making inference of comparison between median and 90 \% percentile
  model, \polya{} tree model tends to smooth the coefficient change
  (-0.41 to -0.24, -3.88 to -4.77), while the 'qreg' method would
  'model' the error with completely different distributions analog to
  asymmetric Laplace distribution.}

\begin{figure}[htbp]
  \begin{minipage}{0.5\linewidth}
    \centerline{\includegraphics[scale=0.4]{../bqrpt/results/hist-res}}
  \end{minipage}
  \begin{minipage}{0.5\linewidth}
    \centerline{\includegraphics[scale=0.4]{../bqrpt/results/density-res}}
  \end{minipage}
  \caption[]{\label{fig:tourpost} Estimated residuals ($r_i = (y_i-
    \bm{x_i'\hat{\beta}})/(\bm{x_i'\hat{\gamma}})$), where
    $\hat{\bm{\beta}}, \hat{\bm{\gamma}}$ are the posterior medians. The left
    figure shows the histogram of the residuals, and the right one
    illustrates the estimated predictive density function, where the
    predictive density function is estimated by averaging predictive
    \polya{} tree distribution density over MCMC iterations.}
\end{figure}

Figure~\ref{fig:tourpost} shows the residuals and posterior mean of the
probability density function of $\epsilon$, $\hat{f}(\epsilon)$. We
can see our approach correctly captures the small minor mode on
the right tail. Thus, if upper quantile of the response is of interest, our
approach would result in more accurate estimation of the quantile regression
parameter than other methods. In addition, the right skewness is also
captured by our approach.

\section{Discussion}
This paper introduced a Bayesian approach for linear quantile
regression model simultaneously by introducing mixture of \polya{}
tree priors and estimating heterogeneity parameters. By marginalizing
the predictive density function of the \polya{} tree distribution,
quantiles of interest can be obtained in closed form by inverting the
predictive cumulative density function. Exact posterior inference can
be made via MCMC. Here, quantile lines cannot cross since quantiles
are estimated through density estimation.  The simulations show our
method performs better than the frequentist approach especially when
the error is multimodal and highly skewed. We also applied and
illustrated our approach on the Tours data exploring the relationship
between quantiles of weight loss and age and race.

Further research includes quantile regression for correlated data by
modelling error as a mixture of multivariate \polya{} tree
distribution and shrinking the heterogeneity coefficients to zero for
increased efficiency (Minzhao: just set $\gamma$ prior narrow and
close to zero ?; {\bf might want spike and slab priors here}). {\it
  Minzhao: Spike and slab priors can accommodate zero-inflated
  situation and specialists' priors to help variable selection.} Our
approach allows for quantile regression with missing data under
ignorability by adding a data augmentation step.  We are exploring
extending our approach to allow for nonignorable missingness ({\bf
  ref. Biometrics paper here}).  Also spatial quantile regression
({\bf ref. Gelfand here}) with \polya{} tree might be a promising area
for further exploration.

\bibliographystyle{plainnat}
% \bibliographystyle{abbrev}
\bibliography{qr-draft-reference}

\appendix
\section{Small Proofs}

Proof for equation (\ref{eq:1}) :
\begin{align*}
  \pr (Y \le \bm{X'\beta}(\tau)) & = \pr \left( \bm{x'\beta} +
    \epsilon
    \le \bm{x'\beta} + F^{-1}_{\epsilon}(\tau) \right) \\
  & = \pr (\epsilon \le F^{-1}_{\epsilon}(\tau)) \\
  & = \tau.
\end{align*}

Proof for equation (\ref{eq:2})
\begin{align*}
  \pr (Y \le \bm{x'\beta}(\tau)) & = \pr \left( \bm{x'\beta} +
    (\bm{x}'\gamma) \epsilon \le \bm{x'\beta} + (\bm{x'\gamma})
    F^{-1}_{\epsilon}(\tau) \right) \\
  & = \pr \left( (\bm{x'\gamma}) \epsilon \le (\bm{x'\gamma})
    F^{-1}_{\epsilon}(\tau)  \right)\\
  & = \pr (\epsilon \le F^{-1}_{\epsilon}(\tau)) \\
  & = \tau .
\end{align*}

\section{Additional Simulation}

Three models with other errors are also tested in simulations to check
performance of our approach. First two models were using a skewed
mixture of normal distribution error (Reich 2009) $\pi N(0,1) +
(1-\pi) N(3,3)$, where $\pi \sim Bern(0.8)$. One of them is
homogeneous and the other is heterogeneous with $\gamma=(1,
-0.5,0.5)$. The third example uses student t distribution with degree
freedom of three. Also same heterogeneity parameters $\gamma=(1,
-0.5,0.5)$ were assigned to this model.

\begin{itemize}
\item [M1] $y_i = 1+x_{i1} + x_{i2} + \epsilon_{i1}$,
\item [M2] $y_i = 1+x_{i1} + x_{i2} + (1 -0.5x_{i1} +
  0.5x_{i2})\epsilon_{i1}$,
\item [M3] $y_i = 1+x_{i1} + x_{i2} + (1 -0.5x_{i1} +
  0.5x_{i2})\epsilon_{i2}$,
\end{itemize}
where $\epsilon_{i1} \sim \pi N(0,1) + (1-\pi) N(3,3)$, $\pi \sim
Bern(0.8)$, and $\epsilon_{i2} \sim t_3$.

\begin{table}[htbp]\label{tab:add}
  \caption[]{ Result for additional simulations: mean squared error
    (reported as 100*average) and standard error (reported as
    100*standard error) for each quantile regression method. 'BQR',
    'PT', 'rq' stand for Reich's Bayesian flexible quantile regression
    method, our approach with \polya{} tree and the frequentist method
    in R package 'quantreg'. }
  \vspace{4mm}

  \centering
  \begin{tabular}[tb]{l|r|r|r|r}
    Model & $\tau$ & BQR         & PT          & rq           \\
    \hline
    M1    & 0.5    & 1.87(0.22)  & 2.24(0.27)  & 2.33(0.32)   \\
          & 0.9    & 8.77(1.01)  & 11.30(1.32) & 17.27(1.81)  \\
    M2    & 0.5    & 7.11(0.92)  & 8.11(0.86)  & 11.64(1.49)  \\
          & 0.9    & 39.41(4.09) & 44.91(4.28) & 103.0(10.52) \\
    M3    & 0.5    & 6.12(0.63)  & 6.93(0.69)  & 7.23(0.69)   \\
          & 0.9    & 18.38(2.06) & 24.16(2.32) & 38.80(4.69)
  \end{tabular}
\end{table}

From table \ref{tab:add}, Reich's method ('BQR') is generally better
than our approach ('PT') and our method beats the traditional
frequentist method 'rq' function in terms of MSE.

\end{document}
%%% Local Variables:
%%% mode: latex
%%% TeX-master: t
%%% End:
