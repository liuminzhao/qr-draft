\section{Multivariate Bayesian Quantile Regression with \polya{} Tree}
\label{sec:multi}

\subsection{Multivariate \polya{} Trees}

Due to the difficulty in definition of quantiles in multivariate case
and wide array of possible partition methods, there is not much literatures
about \polya{} trees priors for multivariate data. \citet{paddock1999,
  paddock2002} extended the univariate \polya{} tree
\citep{lavine1992,lavine1994} to multivariate case in a
q-dimensional hypercube. Partitions are constructed through a series
of binary recursive perpendicular splits of each axis of the
hypercube. \citet{hanson2006} proposed a q-dimensional location-scale
mixture of finite \polya{} trees which is a direct generalization of
the univariate finite location-scale \polya{} tree. \citet{jara2009}
extended the multivariate \polya{} tree prior based on
\citet{hanson2006} with an additional parameter: directional
orthogonal matrix.

Based on \citet{hanson2006} and \citet{jara2009}, we briefly introduce
multivariate \polya{} tree priors as follows: Let $E=\{0,1\}$,
$E^m=\{0,1\}^m$ be m-fold product of $E$, and $\pi_j = \left\{
  B_{\epsilon_{11}\cdots
    \epsilon_{1j};\cdots;\epsilon_{q1}\cdots\epsilon_{qj}};
  \epsilon_{ij}\in E \right\}$ be a level $j$ partition set of
$\Omega$ such that $\pi_{j+1}$ are the $2^q$ finer partitions of
$\pi_j$.

\begin{deff}[Multivariate \polya{} Tree Distribution]
  A q-dimensional random probability measure $G$ is said to have a
  multivariate \polya{} tree distribution with parameters $(\Pi,
  \mathcal{A})$, if there exists nonnegative numbers
  $\mathcal{A}=\left\{ \alpha_{\varepsilon_1;\cdots;\varepsilon_q} ;
    \varepsilon_1, \cdots, \varepsilon_q \in E^j, j=1, \ldots
  \right\}$ (note: $\varepsilon_i$ indicates which position the bin
  takes in level $j$ with respect to $i^{th}$ dimension) and random
  vectors $\mathcal{Y} = \left\{
    Y_{\varepsilon_1;\cdots;\varepsilon_q} ; \varepsilon_1, \cdots,
    \varepsilon_q \in E^j, j=1, \cdots \right\}$, such that the
  following hold:
  \begin{enumerate}
  \item All of the random vectors in $\mathcal{Y}$ are independent,
  \item For $j=1, \ldots$ and for all $\varepsilon_1, \cdots,
    \varepsilon_q \in E^j$,
    $\bm{Y}_{\varepsilon_1;\cdots;\varepsilon_q} \sim
    \mathrm{Dirichlet}\left( \bm{\alpha}_{\varepsilon_1; \cdots;
        \varepsilon_q} \right)$, where
    $\bm{Y}_{\varepsilon_1;\cdots;\varepsilon_q} = \left\{
      y_{\varepsilon_1\epsilon_1; \cdots; \varepsilon_q\epsilon_q};
      \epsilon_1, \cdots, \epsilon_q \in E \right\}$ and
    $\bm{\alpha}_{\varepsilon_1; \cdots; \varepsilon_q} = \left\{
      \alpha_{\varepsilon_1\epsilon_1; \cdots;
        \varepsilon_q\epsilon_q}; \epsilon_1, \cdots, \epsilon_q \in E
    \right\}$,
  \item For every $j=1,2, \cdots$,
    \begin{displaymath}
      G(B_{\epsilon_{11},\cdots,
        \epsilon_{1j};\cdots;\epsilon_{q1}\cdots\epsilon_{qj}}) =
      \prod_{l=1}^j Y_{\epsilon_{11}\cdots \epsilon_{1l}; \cdots ;
        \epsilon_{q1}\cdots\epsilon_{ql}}.
    \end{displaymath}
  \end{enumerate}
\end{deff}

Similar to univariate \polya{} tree, the canonical way of
constructing the partition is based on inverting CDF of the centering
distribution. First, suppose $G_0$ is a univariate cdf and its
corresponding pdf is $g_0(\omega)$. Define $g_0(\bm{\omega}=(\omega_1,
\ldots, \omega_q)) = \prod_{i=1}^q g_0(\omega_i)$. Denote
$\bm{\theta}= (\bm{\mu}_q, \bm{\Sigma}_{q\times q}$ as location-scale
parameters, then a family of location-scale baseline measures for
multivariate \polya{} tree have the following pdf forms
$g_{\bm{\theta}} ( \bm{\omega}) = |\bm{\Sigma}|^{-1/2} g_0 (
\bm{\Sigma}^{-1/2} (\bm{\omega} - \bm{\mu}) ) $.

For baseline measure $g_0(\bm{\omega})$, the partition $\Pi_0^j$ of
$\mathrm{R}^q$ are obtained from cross-products of corresponding
univariate partition sets. Denote
$$B_0( \epsilon_{11}\cdots
\epsilon_{1j};\cdots;\epsilon_{q1}\cdots\epsilon_{qj}) = B_0(e_j(k_1))
\times B_0(e_j(k_2)) \times \cdots \times B_0(e_j(k_q)),$$ where
$B_0(e_j(k))= \left( G_0^{-1}((k-1)2^{-j}), G_0^{-1}(k2^{-j})
\right)$.

Denote $\bm{e}_j(\bm{k})= e_j(k_1); \cdots;e_j(k_q)$, where $\bm{k}=
(k_1, \ldots, k_q)$, then partitions $\Pi_{\theta}^j$ from
location-scale baseline measure family $G_{\bm{\theta}}$ or
$g_{\bm{\theta}}(\bm{\omega})$ are defined as
\begin{displaymath}
  B_{\bm{\theta}}(\bm{e}_j(\bm{k})) = \left\{ \bm{\mu} +
    \bm{\Sigma}^{1/2} \bm{y}; \bm{y} \in B_0(\bm{e}_j(\bm{k})) \right\}.
\end{displaymath}

\citet{jara2009} pointed out that the direction of the sets in
\citet{hanson2006} is completely defined by the decomposition of the
covariance matrix, the unique symmetric square root. As a result, he
introduced another orthogonal matrix as additional parameter to
control the direction of the sets.

Suppose $\bm{\Sigma} = \bm{T'T}$, where $\bm{T}$ is the unique upper
triangular Cholesky matrix, then for any orthogonal matrix $\bm{O}$,
let $\bm{U=OT}$, then $\bm{U}$ is also a square root of
$\bm{\Sigma}$. Therefore, if we put a prior for $\bm{O}$ on the space
of all $q \times q$ orthogonal matrices, then we have a prior on all
possible square roots of $\bm{\Sigma}$, which control the direction of
the partition sets.

Lemma 1 \citep{jara2009} shows it is well
defined. In this way, the location-scale transformation induces
partition sets $B_{\bm{\theta}}(\bm{e}_j(\bm{k})) = \left\{ \bm{\mu} +
  \bm{T'O'} \bm{z}; \bm{z} \in B_0(\bm{e}_j(\bm{k})) \right\}.$ The
Haar measure \citep{halmos1950} provides an easy way to sample
orthogonal matrix $\bm{O}$ uniformly.

\subsection{Multivariate Regression with \polya{} Tree}

In order to address clustered or correlated data, we propose to model
multivariate errors directly instead of adding random effects. We
assume each component of subject's multivariate response can be
affected by covariates on its mean and variance, therefore we propose
a heterogeneous q-dimensional multivariate regression model:
\begin{displaymath}
  \bm{Y}_i = \bm{X}_i \bm{B} + (\bm{X}_i\bm{\Gamma}) \circ \bm{E}_i,
\end{displaymath}
where $\bm{Y}_i = [y_{i1}, \ldots, y_{iq}]^T$, $\bm{X}_i= [x_{i1},
\cdots, x_{ip}]$, $\bm{E}_i =[\epsilon_{i1}, \cdots, \epsilon_{iq}]$,
\begin{displaymath}
  \bm{B}_{p\times q} =
  \begin{bmatrix}
    \beta_{11}& \cdots & \beta_{1q} \\
    \vdots & \ddots & \vdots \\
    \beta_{p1} & \cdots & \beta_{pq}
  \end{bmatrix}
  \quad \text{  and  } \quad
  \bm{Gamma}_{p\times q} =
  \begin{bmatrix}
    \gamma_{11} & \cdots & \gamma_{1q} \\
    \vdots & \ddots & \vdots \\
    \gamma_{p1} & \cdots & \gamma_{pq}
  \end{bmatrix}.
\end{displaymath}
For $n$ subjects and $q$ dimensional responses for each subject, we
have
\begin{align}
  \begin{bmatrix}
    y_{11} &\cdots & y_{1q} \\
    \vdots & \ddots & \vdots \\
    y_{n1} & \cdots & y_{nq}
  \end{bmatrix}_{n\times q} & =
  \begin{bmatrix}
    x_{11} &\cdots & x_{1p} \\
    \vdots & \ddots & \vdots \\
    x_{n1} & \cdots & x_{np}
  \end{bmatrix}_{n\times p}
  \begin{bmatrix}
    \beta_{11} &\cdots & \beta_{1q} \\
    \vdots & \ddots & \vdots \\
    \beta_{p1} & \cdots & \beta_{pq}
  \end{bmatrix}_{p\times q} \nonumber\\
  & \quad + \left( \begin{bmatrix}
      x_{11} &\cdots & x_{1p} \\
      \vdots & \ddots & \vdots \\
      x_{n1} & \cdots & x_{np}
    \end{bmatrix}
    \begin{bmatrix}
      \gamma_{11} &\cdots & \gamma_{1q} \\
      \vdots & \ddots & \vdots \\
      \gamma_{p1} & \cdots & \gamma_{pq}
    \end{bmatrix}
  \right) \circ
  \begin{bmatrix}
    \epsilon_{11} &\cdots & \epsilon_{1q} \\
    \vdots & \ddots & \vdots \\
    \epsilon_{n1} & \cdots & \epsilon_{nq}
  \end{bmatrix}_{n\times q}
\end{align}
in which $\circ$ is Hadamard product (entrywise product). We assign a
multivariate \polya{} tree prior on the error:
\begin{align*}
  &\bm{E}_i = [\epsilon_{i1}, \cdots, \epsilon_{iq}]^T
  \stackrel{\text{i.i.d}}{\sim} G_{\bm{\theta}} \\
  &G_{\bm{\theta}} | \bm{\mu, \Sigma, O} \sim PT \left( \Pi^{\bm{\mu,
        \Sigma, O}}, \mathcal{A} \right).
\end{align*}
In order to not confound the $\bm{\beta}$ estimates, we set $\bm{\mu}
= \bm{0}$ and the medians for each component of $G_{\bm{\theta}}$ are
fixed at 0. For the heterogeneity parameters $\gamma_{ij}$, for the
same reason, we restrict $\gamma_{1j}=1$ and for all
$\bm{\gamma_{.j}}, \bm{x}_{i.}$, $\bm{x_{i.}\gamma_{.j}}>0$ for all
$i,j$.

Analogous to the univariate quantile regression case, the posterior
$\tau$th quantile regression coefficient for component of response can
be obtained from posterior estimates by
\begin{equation}\label{eq:mul}
  \bm{\beta^{(i)}}(\tau) = \bm{\beta^{(i)}} + \bm{\gamma^{(i)}}F^{-1}_{\epsilon^{(i)}}(\tau),
\end{equation}
where $\bm{\beta^{(i)}}$ and $\bm{\gamma^{(i)}}$ is the $i^{th}$
column of $\bm{B}$ and $\bm{\Gamma}$, and $F^{-1}_{\epsilon^{(i)}}$ is
the inverse marginal CDF of $i^{th}$ component of $q$-dim error, which
can be calculated in the same way as in univariate \polya{} tree after
collapsing multivariate residuals.

\subsection{Comparison to Reich}
\citet{reich2010} proposed a flexible Bayesian approach dealing with
clustered data based on both conditional and marginal models. They
added a random effect term to this flexible Bayesian model to address
compound symmetric correlation struction. However, the assumption for
correlation structure is restrictive in many settings. Our proposed
method deals with any correlation structure since \polya{} tree can
capture the error distribution after training through enough data
observations. More specifically, Reich's method restricts the
correlation to be positive and constant across components, while the
quantile regression with \polya{} trees prior works well for negative
correlation, serial correlation or any other scenarios.

In addition, Reich's approach can only make inference for the
quantiles of the dependent variables itself, rather than linear
combinations of components in a multivariate dependent random
vector. For example in longitudinal data, quantiles of measurement
differences between time points are often of interest, e.g. in data
example.  However our method can easily handle such inferences through
posterior sampling of $y_3-y_0$ and using \polya{} tree technique to
draw posterior quantiles.


%%% Local Variables:
%%% mode: latex
%%% TeX-master: t
%%% End:
